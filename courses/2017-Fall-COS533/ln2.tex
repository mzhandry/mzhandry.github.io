\documentclass[12pt]{article}
\usepackage
[pdftex,pagebackref,colorlinks=true,pdfpagemode=UseNone,urlcolor=blue,linkcolor=blue,citecolor=blue,pdfstartview=FitH]{hyperref}

\usepackage{amsmath,amsfonts}
\usepackage{amsthm} 
\usepackage{graphicx}

\bibliographystyle{alpha}

\setlength{\oddsidemargin}{0pt}
\setlength{\evensidemargin}{0pt}
\setlength{\textwidth}{6.0in}
\setlength{\topmargin}{0in}
\setlength{\textheight}{8.5in}

\setlength{\parindent}{0in}
\setlength{\parskip}{5px}




%%%%%%%%% Macros


\newtheorem{theorem}{Theorem}
\newtheorem{lemma}[theorem]{Lemma}
\newtheorem{corollary}[theorem]{Corollary}
\newtheorem{definition}[theorem]{Definition}
\newtheorem{example}[theorem]{Example}
\newtheorem{remark}[theorem]{Remark}
\newtheorem{construction}[theorem]{Construction}


%%%%%%%%% Crypto Primitives

\newcommand{\obf}{{\sf{Obf}}}
\newcommand{\iO}{{\sf{iO}}}
\newcommand{\VBBO}{{\sf{VBBO}}}
\newcommand{\diO}{{\sf{diO}}}
\newcommand{\owf}{{\sf{OWF}}}
\newcommand{\prg}{{\sf{PRG}}}
\newcommand{\prf}{{\sf{PRF}}}
\newcommand{\gen}{{\sf{Gen}}}
\newcommand{\enc}{{\sf{Enc}}}
\newcommand{\dec}{{\sf{Dec}}}
\newcommand{\sign}{{\sf{Sign}}}
\newcommand{\ver}{{\sf{Ver}}}

%%%%%%%%% Extra Algorithms (added by Fermi)
\newcommand{\params}{{\sf{params}}}
\newcommand{\publish}{{\sf{Publish}}}
\newcommand{\setup}{{\sf{Setup}}}
\newcommand{\extract}{{\sf{Extract}}}
\newcommand{\keygen}{{\sf{KeyGen}}}


%%%%%%%%% Keys
\newcommand{\dk}{{\sf dk}}
\newcommand{\ek}{{\sf ek}}
\newcommand{\sk}{{\sf sk}}
\newcommand{\vk}{{\sf vk}}
\newcommand{\mpk}{{\sf mpk}}
\newcommand{\msk}{{\sf msk}}
\newcommand{\id}{{\sf id}}



%%%%%%%%% Functions
\newcommand{\polylog}{{\sf polylog}}
\newcommand{\poly}{{\sf poly}}
\newcommand{\negl}{{\sf negl}}


%%%%%%%%% To typeset the header

\def\lecturer{Mark Zhandry}
\def\coursenum{COS 597A}
\def\coursename{Quantum Cryptography}

\newlength{\tpush}
\setlength{\tpush}{2\headheight}
\addtolength{\tpush}{\headsep}


\newcommand{\lecturenotes}[3]{\noindent\vspace*{-\tpush}\newline\parbox{\textwidth}
	{\coursenum : \coursename \hfill Princeton University \newline
		Lecture #1 (#2) \newline
		Lecturer: \lecturer \hfill Scribe: #3 \newline
		\mbox{}\hrulefill\mbox{}}\vspace*{1ex}\mbox{}\newline
	\bigskip
	\begin{center}{\Large\bf  Notes for Lecture #1}\end{center}
	\bigskip}


\newcommand{\homeworks}[2]{\noindent\vspace*{-\tpush}\newline\parbox{\textwidth}
	{\coursenum : \coursename \hfill Princeton University \newline
		Homework #1 \hfill Due: #2 \newline
		\mbox{}\hrulefill\mbox{}}\vspace*{1ex}\mbox{}\newline
	\bigskip
	\begin{center}{\Large\bf  Homework #1}\end{center}
	\bigskip}













\begin{document}

\lecturenotes{2}{September 18, 2017}{Mark Zhandry}

\section{Last Time}

Last time, we defined formally what an encryption scheme is.  A (symmetric key or secret key) encryption scheme consists of two algorithms $(\enc,\dec)$.  $\enc$ is a PPT algorithm that takes as input a key and a plaintext, and outputs a ciphertext.  $\dec$ is deterministic polynomial time, takes as input a key and a ciphertext, and outputs a plaintext.  For correctness, we require that when used with the same key, $\dec$ inverts $\enc$.  More precisely, for all messages $m$,
\[\Pr[\dec(k,\enc(k,m))=m,k\stackrel{\$}{\gets}\{0,1\}^\lambda]=1\]
Here, the probability is taken over a random $k$, and any random coins chosen by $\enc$.  

For security, let $A$ be an adversary.  Let $\text{\sf IND-CPA-EXP}_b(A,\lambda)$ be the following experiment on $A$, parameterized by a bit $b$:
\begin{enumerate}
	\item $A$ interacts with a challenger, denoted $Ch$.
	\item At first, $Ch$ chooses a random key $k\stackrel{\$}{\gets}\{0,1\}^\lambda$
	\item Next, $A$ sends the challenger two messages $m_0,m_1$.  $Ch$ selects and encrypts $m_b$: $c\gets\enc(k,m_b)$.  Then $Ch$ sends $c$ back to $A$.
	\item $A$ can repeat step 3 as many times as it wishes.  We will charge $A$ one unit of time for every time it repeats step 3.
	\item Finally, $A$ outputs a guess $b'$ for $b$.  $b'$ is the output of $\text{\sf IND-CPA-EXP}_b(A,\lambda)$
\end{enumerate}

Here, ${\sf IND}$ refers to indistinguishability, meaning that the adversary is trying to distinguish between two experiments, $b=0$ and $b=1$.  ${\sf CPA}$ stands for ``chosen plaintext attack''.  This refers to the fact that the adversary is able to choose the plaintexts that get encrypted.

\begin{definition} An encryption scheme $(\enc,\dec)$ is \emph{{\sf IND-CPA} secure} (in words, \emph{indistinguishable under a chosen plaintext attack}) if, for all PPT adversaries $A$, there exists a negligible function $\epsilon$ such that
	\[|\;\Pr[1\gets\text{\sf IND-CPA-EXP}_0(A,\lambda)]-\Pr[1\gets\text{\sf IND-CPA-EXP}_1(A,\lambda)]\;|<\epsilon(\lambda)\]
\end{definition}

We will often simply call such a scheme ``CPA secure''.  Intuitively, this definition means that any guess $b'$ the adversary makes is more or less independent of the actual bit $b$, since the probabilities for any guess under the two experiments are extremely close.


\section{This Time}

Starting today, and for the next couple lectures, we will show how to construct encryption and other cryptographic applications from weaker tools.  In particular, we will show:
\begin{enumerate}
	\item PRFs (pseudorandom functions) $\rightarrow$ CPA-secure secret key encryption
	\item PRGs (pseudorandom generators) $\rightarrow$ PRFs
	\item OWPs (one-way permutations) $\rightarrow$ PRGs
	\item PRFs $\rightarrow$ MACs (message authentication codes)
	\item MACs + CPA-secure secret key encryption $\rightarrow$ CCA-secure secret key encryption
	\item PRFs + UOWHFs (universal one-way hash functions) $\rightarrow$ digial signatures (aka public key MACs)
\end{enumerate}
Additionally, it is known how to improve step 3 to ``OWF (one-way functions) $\rightarrow$ PRGs'' and to build show that ``OWF $\rightarrow$ UOWHFs''.  Therefore, all of the crypto primitives above can be build from one-way functions.

Today, we will show the first two steps, namely how to build encryption from PRFs and PRFs from PRGs.


\section{PRFs}

A PRF is a keyed function that looks like a random function if you never get to see the key.  That is $\prf$ is a deterministic polynomial time computable function $\prf:\{0,1\}^\lambda\times\{0,1\}^{n(\lambda)}\rightarrow\{0,1\}^{m(\lambda)}$, with the following security property.  

Let $A$ be an adversary.  Let $\text{\sf PRF-EXP}_b(A,\lambda)$ be the following experiment on $A$, parameterized by a bit $b$:
\begin{enumerate}
	\item $A$ interacts with a challenger, denoted $Ch$.
	\item At first, if $b=0$, $Ch$ chooses a random key $k\stackrel{\$}{\gets}\{0,1\}^\lambda$.  If $b=1$, $Ch$ initializes an empty list $L$.
	\item Next, $A$ sends the challenger an input $x\in\{0,1\}^{n(\lambda)}$.  $Ch$ responds as follows
	\begin{itemize}
		\item If $b=0$, $Ch$ responds with $y\gets\prf(k,x)$.
		\item If $b=1$, $Ch$ looks for a pair $(x,y)$ in $L$.  If it finds an $(x,y)$, it responds with $y$.  Otherwise, it generates a random $y$ and adds the pair $(x,y)$ to $L$.  Then it responds with $y$.
	\end{itemize}
	\item $A$ can repeat step 3 as many times as it wishes.  We will charge $A$ one unit of time for every time it repeats step 3.
	\item Finally, $A$ outputs a guess $b'$ for $b$.  $b'$ is the output of $\text{\sf PRF-EXP}_b(A,\lambda)$
\end{enumerate}

Notice that in the $b=1$ case, $Ch$ is effectively providing $A$ with a truly random function $O$ where all outputs are chosen independently and uniformly at random.  In the $b=0$ case, $Ch$ is providing $A$ with the PRF on a random key $k$.  $A$'s goal is to distinguish the two cases.

\begin{definition} An PRF $\prf$ \emph{secure} if, for all PPT adversaries $A$, there exists a negligible function $\epsilon$ such that
	\[|\;\Pr[1\gets\text{\sf PRF-EXP}_0(A,\lambda)]-\Pr[1\gets\text{\sf PRF-EXP}_1(A,\lambda)]\;|<\epsilon(\lambda)\]
\end{definition}


\section{CPA-secure secret key encryption from PRGs}

Let $\prf$ be a pseudorandom function $\prf:\{0,1\}^\lambda\times\{0,1\}^{n(\lambda)}\rightarrow\{0,1\}^{m(\lambda)}$.  

Our scheme will be the following:
\begin{itemize}
	\item $\enc(k,m)$ for $k\in\{0,1\}^\lambda$ and $m\in\{0,1\}^{m(\lambda)}$ does the following.  First it chooses a random $r\in\{0,1\}^\lambda$, and computes $c\gets\prf(k,r)\oplus m$.  It outputs $(r,c)$
	\item $\dec(k,(r,c))$ computes $m\gets\prf(k,r)\oplus c$
\end{itemize}
Correctness is straightforward, since $\dec$ computes $\prf(k,r)\oplus c=\prf(k,r)\oplus(\prf(k,r)\oplus m)=m$.

\begin{theorem} If $\prf$ is a secure PRF and $2^{n(\lambda)}$ is super polynomial, then $(\enc,\dec)$ is CPA secure.
\end{theorem}

Proofs in cryptography are usually proofs by contradiction: we assume an adversary violating the security of one primitive (in our case, an encryption scheme), and derive from it an adversary for some starting primitive (in our case, a PRF).
	
We will also introduce one of the standard proof techniques in cryptography, a hybrid argument.  Assume toward contradition that there is an adversary $A$ and a non-negligible function $\epsilon$ such that
\[|\;\Pr[1\gets\text{\sf IND-CPA-EXP}_0(A,\lambda)]-\Pr[1\gets\text{\sf IND-CPA-EXP}_1(A,\lambda)]\;|\geq \epsilon(\lambda)\]

We will define several ``hybrid'' games, where the first is $\text{\sf IND-CPA-EXP}_0(A,\lambda)$, and the last is $\text{\sf IND-CPA-EXP}_0(A,\lambda)$.  By our assumption, we know that $A$ distinguishes the first and last hybrid.  Therefore, it must also distinguish some pair of adjacent intermediate hybrids.  We will use such a distinguishing advantage to break the security of the PRF.
\begin{itemize}
	\item {\bf Hybrid 0} is identical to $\text{\sf IND-CPA-EXP}_0(A,\lambda)$.  Substituting in to the experiment our construction, the experiment works as follows:
	\begin{enumerate}
		\item At first, $Ch$ chooses a random key $k\stackrel{\$}{\gets}\{0,1\}^\lambda$
		\item Next, $A$ sends the challenger two messages $m_0,m_1$.  $Ch$ chooses a random $r$ in $\{0,1\}^{n(\lambda)}$, and computes $y\gets\prf(k,r)$.  Then $Ch$ sends $(r,y\oplus m_0)$ back to $A$.  $A$ can repeat this step as many times as it wishes.
s	\end{enumerate}
	\item {\bf Hybrid 1} is the following modifications to the $\text{\sf IND-CPA-EXP}_0(A,\lambda)$ experiment.  
	\begin{itemize}
		\item Instead of generating a random key $k$, $Ch$ initializes an empty list $L$.
		\item Instead of computing $y\gets\prf(k,r)$, $Ch$ looks for $(r,y)$ in $L$, using $y$ if found.  Otherwise, it generates a fresh random $y$, and adds $(r,y)$ to $L$.
	\end{itemize} 
	\item {\bf Hybrid 2} is the same as {\bf Hybrid 1}, except that $Ch$ sends $(r,y\oplus m_1)$ back to $A$.
	\item {\bf Hybrid 3} is the same as {\bf Hybrid 2}, except that $Ch$ goes back to choosing a random key $k$ and setting $y\gets\prf(k,r)$.  Notice that {\bf Hybrid 3} is identical to $\text{\sf IND-CPA-EXP}_1(A,\lambda)$
\end{itemize}

Now, by our assumption that $(\enc,\dec)$ is insecure and the triangle inequality, we have that there must exist an $i\in\{0,1,2\}$ such that 
\[|\;\Pr[1\gets\text{Hybrid}_i(A,\lambda)]-\Pr[1\gets\text{Hybrid}_{i+1}(A,\lambda)]\;|\geq \frac{1}{3}\epsilon(\lambda)\]
We now consider the three cases:
\begin{itemize}
	\item $i=0$.  Notice that the only difference between ${\bf Hybrid 0}$ and ${\bf Hybrid 1}$ is that in {\bf Hybrid 0}, $y$ is set to $\prf(k,r)$, whereas in {\bf Hybrid 1}, $y$ is set to random.  It is straightforward to construct a PRF adversary $B$ which distinguishes $\prf$ from random with advantage $\epsilon(\lambda)/3$.  $B$ works as follows: it simulates $A$, playing the role of CPA-security challenger to $A$.  Whenever $A$ makes a query $(m_0,m_1)$, $B$ chooses a random $r$, and queries its own PRF challenger on $r$, obtaining $y$.  Then it responds to $A$ with $(r,y\oplus m_0)$.  Finally, when $A$ outputs a bit $b'$, $B$ outputs $b'$.
	
	In $\text{\sf PRF-EXP}_0(A,\lambda)]$, $B$ successfully simulates the view of $A$ in {\bf Hybrid 0}.  Similarly, in $\text{\sf PRF-EXP}_1(A,\lambda)]$, $B$ successfully simulates the view of $A$ in {\bf Hybrid 1}.  Therefore, $B$'s advantage is the same as $A$'s in distinguishing these two hybrids, namely $\epsilon(\lambda)/3$.  This is non-negligible, a contradiction to the assumed security of $\prf$.
	
	\item $i=1$.  In {\bf Hybrid 1} and {\bf Hybrid 2}, $y$ is set to random; the only difference is that it is XORed with $m_0$ or $m_1$ before responding to $A$.  However, a random string XORed with anything is still random, so $A$ essentially receives random strings in both hybrids.
	
	The only potential problem if the same $r$ is used to encrypt in two different queries.  In this case, the response to each query is random, but the two responses are correlated.  
	
	Such collisions in $r$ are, however, not a common occurrence: the probability any two queries collide is $2^-n(\lambda)$.  The probability that some pair of queries collide is therefore at most $q^2\times 2^{-n(\lambda)}$ where $q$ is the number of queries.  Recall that $q$ is a polynomial, and that $2^{-n(\lambda)}$ is negligible.  Since a negligible function times a polynomial is still negligible, we have that the probability of a collision is negligible.  Therefore, $A$'s distinguishing advantage $\epsilon(\lambda)/3$ must be negligible, a contradiction.
	\item $i=2$.  This is handled identically to $i=0$, except that $B$ encrypts $m_1$ instead of $m_0$.
\end{itemize}

Therefore, in any of the three cases, we reach a contradiction.  Therefore, our assumed adversary $A$ could not possibly exist.  This completes the proof of security.



\section{PRGs}

Next, we turn to constructing PRFs from a weaker object called a pseudorandom generator, or PRG.  A PRG is a deterministic polynomial time function $G:\{0,1\}^\lambda\rightarrow\{0,1\}^{\lambda+s(\lambda)}$ for some $s(\lambda)>1$.  This means that $G$ expands its input.  For security, we ask that outputs of $G$ look as if they were random.

\begin{definition} A function $G$ is a secure PRG if, for all PPT adversaries $A$, there exists a negligible function $\epsilon$ such that
	\[|\;\Pr[A(G(x))=1:x\stackrel{\$}{\gets}\{0,1\}^\lambda]-\Pr[A(y)=1:y\stackrel{\$}{\gets}\{0,1\}^{\lambda+s(\lambda)}]\;|<\epsilon(\lambda)\]
\end{definition}

Notice that $G$ can only take on at most $2^\lambda$ outputs, smaller than the $2^{\lambda+s(\lambda)}$ points in the co-domain.  Therefore, it is impossible for $G$'s outputs to be truly random.  Nonetheless, PRG security says that the outputs \emph{look} random to any polynomial-time adversary.

\section{PRFs from PRGs}

We will not formally define the actual PRF algorithm, but instead will describe it in words.  We will assume $G$ is length-doubling, meaning $s(\lambda)=\lambda$.  First, let's forget efficiency for the moment.  The PRF $\prf$ works as follows.  It takes the key $k\in\{0,1\}^\lambda$, and applies $G$.  The result is a $2\lambda$-bit string.  $\prf$ splits the string in half into two $\lambda$-bit strings.  Then it applies $G$ separately to both halves, obtaining a $4\lambda$-bit string.  $\prf$ splits this into 4 $\lambda$-bits strings, and applies $G$ to each string.  It continues in this way for $n(\lambda)$ steps, for any desired polynomial $\lambda$.

The result is a $2^{n(\lambda)}\times\lambda$-bit string.  $\prf$ will interpret this string as $2^{n(\lambda)}$ separate $\lambda$-bit strings.  The output of $\prf$ on input $x$ with be the $x$th string in this list.  Therefore, $\prf$ has inputs of length $n(\lambda)$ and outputs of length $\lambda$.

It will be useful to think of the PRF computation as a tree: at the root is the PRF key $k$.  The children of a node containing $x$ are the first and second half of $G(x)$.  The tree has $n+1$ levels and $2^n$ leaves.  The leaves are the outputs of the PRF.

As described, $\prf$ runs in time roughly $2^{n(\lambda)}$, which is exponential.

\begin{center}
Question: How to compute each block locally in time polynomial in $n$, without computing the entire list of outputs?
\end{center}

\begin{theorem}If $G$ is a secure PRG, then the construction above is a secure PRF\end{theorem}

As before, we will prove this theorem by a hybrid argument.  Assume toward contradiction that there is an adversary $A$ and a non-negligle function $\epsilon$ such that 
\[|\;\Pr[1\gets\text{\sf PRF-EXP}_0(A,\lambda)]-\Pr[1\gets\text{\sf PRF-EXP}_1(A,\lambda)]\;|\geq \epsilon(\lambda)\]

Define {\bf Hybrid $i$} for $i\in\{0,\dots,n\}$ as follows.  {\bf Hybrid 0} is just $\text{\sf PRF-EXP}_0(A,\lambda)$, where $A$ interacts with $\prf$.  In {\bf Hybrid 1}, we slightly modify the experiment.  Instead of choosing a random key $k\in\{0,1\}^\lambda$ and placing it at root of the PRF tree, $Ch$ chooses two random strings $x_0,x_1$, and places them at level 1 of the tree (here we zero index the tree levels).  It computes all nodes below the level 1 just as in the PRF: the level 2 is obtained by applying $G$ to the elements of level 1, etc.  

{\bf Hybrid 2} is defined analogously: choose random $x_{00},x_{01},x_{10},x_{11}$, place them in level 2 of the tree, and generate the nodes levels 3 through $n$  as before.  Similarly define the remaining hybrids.

Note that in {\bf Hybrid $n$}, all the leaves of the tree have uniformly random elements; this corresponds to $A$ interacting with a truly random function, namely $\text{\sf PRF-EXP}_1(A,\lambda)$.  Therefore, there exists an $i\in\{0,\dots,n-1\}$ such that 
\[|\;\Pr[1\gets\text{Hybrid}_{i}(A,\lambda)]-\Pr[1\gets\text{\sf Hybrid}_{i+1}(A,\lambda)]\;|\geq \frac{\epsilon(\lambda)}{n(\lambda)}\]

Notice that $\epsilon(\lambda)/n(\lambda)$ is non-negligible, since $n$ is a polynomial.  However, this still does not give us a PRG adversary.  To finally get a PRG adversary, we introduce another sequence of hybrids {\bf Hybrid $i.j$} for $j\in\{0,\dots,2^i\}$.  {\bf Hybrid $i.0$} is taken to be {\bf Hybrid $i$}, namely we fill level $i$ with uniformly random elements, and derive levels $i+1$ through $n$ using $G$.  In {\bf Hybrid $i.j$}, we fill the first $2j$ elements of level $(i+1)$ with random elements, and the last $2^i-j$ elements of level $i$ with random elements.  For nodes $2j+1,\dots,2^(i+1)$ of level $(i+1)$ are derived from the parents using $G$.  The rest of the nodes in levels $i+2$ up to $n$ are derived as before using $G$.  Notice that {\bf Hybrid $i,2^i$} is identical to {\bf Hybrid $i+1$} since we are filling level $i+1$ with random elements.  Therefore, similar to above, we find that there is a $j$ such that

\[|\;\Pr[1\gets\text{Hybrid}_{i.j}(A,\lambda)]-\Pr[1\gets\text{\sf Hybrid}_{i.(j+1)}(A,\lambda)]\;|\geq \frac{\epsilon(\lambda)}{n(\lambda)2^i}\]

At this point, the views of $A$ in these two hybrids differ only by a single node where $G(x)$ was replaced with random.  Hence, it is straightforward to construct a PRG adversary $B$ from $A$.  Roughly, $B$, on input $y$, chooses random elements to fill the first nodes $1$ through $j$ of level $i$ and the nodes $2j+3$ through $2^{i+1}$ of level $i+1$.  For nodes $2j+1,2j+2$ of level $i+1$, it puts in $y$.  Then is simulates $A$ with access to the tree derived from these nodes, and outputs the result of $A$.  If $y=G(x)$ for a random $x$, then the view of $A$ is identical to {\bf Hybrid $i.j$}, and if $y$ is random, then the view of $A$ is identicalto {\bf Hybrid $i.(j+1)$}.  Therefore, $B$'s distinguishing advantage is the same as the distinguishing advantage of $A$ in these two hybrids, so we have that 
\[|\;\Pr[1\gets B(G(x)):x\stackrel{\$}{\gets}\{0,1\}^\lambda]-\Pr[1\gets B(y):y\stackrel{\$}{\gets}\{0,1\}^{2\lambda}]\;|\geq \frac{\epsilon(\lambda)}{n(\lambda)2^i}\]

Unfortunately, $i$ could be as large as $n$, and so the value on the right side could be exponentially small.  Therefore, we do not necessarily get a contradiction.  A more clever argument is therefore required.

The key insight is to remember that $A$ is an efficient adversary, and can therefore only make a polynomial number $q(\lambda)$ of queries.  In the PRF tree, this means that $A$ only gets to see $q$ of the leaves of the tree.  If we trace these leaves up to level $i$, we see that the view of $A$ only depends on at most $q$ nodes in level $i$.  For nodes $j$ that $A$ does not depend on, the hybrids {\bf Hybrid $i.j$} and {\bf Hybrid $i.(j+1)$} are actually perfectly indistinguishable, since $A$ never even sees the nodes that depend on the change between the hybrids.

This means we can really skip all but $q$ of the hybrids, obtaining an adversary $B$ such that 
\[|\;\Pr[1\gets B(G(x)):x\stackrel{\$}{\gets}\{0,1\}^\lambda]-\Pr[1\gets B(y):y\stackrel{\$}{\gets}\{0,1\}^{2\lambda}]\;|\geq \frac{\epsilon(\lambda)}{n(\lambda)q(\lambda)}\]

Note that the description above is not quite accurate, and basically assumed that $A$ queries on a fixed set of nodes that do not depend on the results of previous queries.  However, with a little care, it is possible to make the proof work even if $A$ makes adaptive queries based on previous responses.

\section{Extending the Length of PRGs}

Above, we assume that $G$ was length-doubling, in other words $s(\lambda)=\lambda$.  Here, we briefly explain how to build such a $G$ from one where $s(\lambda)=1$.  

The idea is similar to above, but we use a chain instead of a tree.  Let $G:\{0,1\}^\lambda\rightarrow\{0,1\}^{\lambda+1}$.  Construct $G':\{0,1\}^\lambda\rightarrow\{0,1\}^{2\lambda}$ as follows.  On input $x$, run $G$, and write its output as $(x_1,b_1)$ for $x_1\in\{0,1\}^\lambda$ and $b_1\in\{0,1\}$.  Then apply $G$ again, this time to $x_1$, obtaining $x_2,b_2$.  Repeat this process $\lambda$ times until you have $x_\lambda,b_1,\dots,b_\lambda$.  Output these as the output of $G$.  Security can be proved through a similar (but simpler) proof as above.

\end{document}
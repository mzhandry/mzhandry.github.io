\documentclass{article}
\usepackage{fullpage}
\usepackage{amsfonts,amsthm,amsmath}
\usepackage{graphicx}
\usepackage[width=0.75\textwidth]{caption}
\usepackage{subcaption}
\usepackage{hyperref}


\newtheorem{theorem}{Theorem}
\newtheorem{remark}[theorem]{Remark}
\newtheorem{definition}[theorem]{Definition}

\newcommand{\C}{{\mathbb{C}}}
\newcommand{\N}{{\mathbb{N}}}
\newcommand{\R}{{\mathbb{R}}}
\newcommand{\Z}{{\mathbb{Z}}}

\newcommand{\As}{{\mathcal{A}}}
\newcommand{\Bs}{{\mathcal{B}}}
\newcommand{\Cs}{{\mathcal{C}}}
\newcommand{\Ds}{{\mathcal{D}}}
\newcommand{\Es}{{\mathcal{E}}}
\newcommand{\Fs}{{\mathcal{F}}}
\newcommand{\Gs}{{\mathcal{G}}}
\newcommand{\Hs}{{\mathcal{H}}}
\newcommand{\Is}{{\mathcal{I}}}
\newcommand{\Js}{{\mathcal{J}}}
\newcommand{\Ks}{{\mathcal{K}}}
\newcommand{\Ls}{{\mathcal{L}}}
\newcommand{\Ms}{{\mathcal{M}}}
\newcommand{\Ns}{{\mathcal{N}}}
\newcommand{\Os}{{\mathcal{O}}}
\newcommand{\Ps}{{\mathcal{P}}}
\newcommand{\Qs}{{\mathcal{Q}}}
\newcommand{\Rs}{{\mathcal{R}}}
\newcommand{\Ss}{{\mathcal{S}}}
\newcommand{\Ts}{{\mathcal{T}}}
\newcommand{\Us}{{\mathcal{U}}}
\newcommand{\Vs}{{\mathcal{V}}}
\newcommand{\Ws}{{\mathcal{W}}}
\newcommand{\Xs}{{\mathcal{X}}}
\newcommand{\Ys}{{\mathcal{Y}}}
\newcommand{\Zs}{{\mathcal{Z}}}

\newcommand{\Am}{{\mathbf{A}}}
\newcommand{\Bm}{{\mathbf{B}}}
\newcommand{\Cm}{{\mathbf{C}}}
\newcommand{\Dm}{{\mathbf{D}}}
\newcommand{\Em}{{\mathbf{E}}}
\newcommand{\Fm}{{\mathbf{F}}}
\newcommand{\Gm}{{\mathbf{G}}}
\newcommand{\Hm}{{\mathbf{H}}}
\newcommand{\Jm}{{\mathbf{J}}}
\newcommand{\Km}{{\mathbf{K}}}
\newcommand{\Lm}{{\mathbf{L}}}
\newcommand{\Mm}{{\mathbf{M}}}
\newcommand{\Nm}{{\mathbf{N}}}
\newcommand{\Om}{{\mathbf{O}}}
\newcommand{\Pm}{{\mathbf{P}}}
\newcommand{\Qm}{{\mathbf{Q}}}
\newcommand{\Rm}{{\mathbf{R}}}
\newcommand{\Sm}{{\mathbf{S}}}
\newcommand{\Tm}{{\mathbf{T}}}
\newcommand{\Um}{{\mathbf{U}}}
\newcommand{\Vm}{{\mathbf{V}}}
\newcommand{\Wm}{{\mathbf{W}}}
\newcommand{\Xm}{{\mathbf{X}}}
\newcommand{\Ym}{{\mathbf{Y}}}
\newcommand{\Zm}{{\mathbf{Z}}}

\newcommand{\Id}{{\mathbf{I}}}

\newcommand{\av}{{\mathbf{a}}}
\newcommand{\bv}{{\mathbf{b}}}
\newcommand{\cv}{{\mathbf{c}}}
\newcommand{\dv}{{\mathbf{d}}}
\newcommand{\ev}{{\mathbf{e}}}
\newcommand{\fv}{{\mathbf{f}}}
\newcommand{\gv}{{\mathbf{g}}}
\newcommand{\hv}{{\mathbf{h}}}
\newcommand{\iv}{{\mathbf{i}}}
\newcommand{\jv}{{\mathbf{j}}}
\newcommand{\kv}{{\mathbf{k}}}
\newcommand{\lv}{{\mathbf{l}}}
\newcommand{\mv}{{\mathbf{m}}}
\newcommand{\nv}{{\mathbf{n}}}
\newcommand{\ov}{{\mathbf{o}}}
\newcommand{\pv}{{\mathbf{p}}}
\newcommand{\qv}{{\mathbf{q}}}
\newcommand{\rv}{{\mathbf{r}}}
\newcommand{\sv}{{\mathbf{s}}}
\newcommand{\tv}{{\mathbf{t}}}
\newcommand{\uv}{{\mathbf{u}}}
\newcommand{\vv}{{\mathbf{v}}}
\newcommand{\wv}{{\mathbf{w}}}
\newcommand{\xv}{{\mathbf{x}}}
\newcommand{\yv}{{\mathbf{y}}}
\newcommand{\zv}{{\mathbf{z}}}

\newcommand{\AND}{{\sf AND}}
\newcommand{\OR}{{\sf OR}}
\newcommand{\NOT}{{\sf NOT}}
\newcommand{\CNOT}{{\sf CNOT}}
\newcommand{\CCNOT}{{\sf CCNOT}}


\newcommand{\poly}{{\sf poly}}
\newcommand{\negl}{{\sf negl}}

\newcommand{\gen}{{\sf Gen}}
\newcommand{\enc}{{\sf Enc}}
\newcommand{\dec}{{\sf Dec}}
\newcommand{\sign}{{\sf Sign}}
\newcommand{\ver}{{\sf Ver}}
\newcommand{\mint}{{\sf Mint}}


\newcommand{\sk}{{\sf sk}}
\newcommand{\pk}{{\sf pk}}

\title{CS 258: Quantum Cryptography (Fall 2025)\\ Homework 5 (100 points)}
\author{}
\date{}

\begin{document}

\maketitle

\section{Problem 1 (30 points)}

Consider a distribution over quantum states, where $|\psi_i\rangle$ is sampled with probability $p_i$. Let $\rho=\sum_i p_i |\psi_i\rangle\langle\psi_i|$ be the resulting density matrix.

\begin{itemize}
    \item {\bf Part (a). 10 points.} Let $U$ be a unitary, and consider computing $|\phi_i\rangle=U|\psi_i\rangle$. Taking the probability over $i$, this gives a new mixed state described by density matrix $\rho_a$. Show that $\rho_a=U\rho U^\dagger$.
    \item {\bf Part (b). 10 points.} For any $\rho$, consider measuring in the computational basis. Show that the probability of a measurement outcome $x$ is given by $\langle x|\rho|x\rangle$.
    \item {\bf Part (c). 10 points.} Measuring in the computational basis gives $x$ with some probability (as computed in Part (b)), and the post-measurement state is then $|x\rangle$. This gives a new probability distribution over quantum states, which is described by a density matrix $\rho_c$. Show that $\rho_c$ is a diagonal matrix obtained from $\rho$ by erasing all the off-diagonal entries.
\end{itemize}



\section{Problem 2 (30 points)}

For two classical probability distributions $D_0,D_1$ their distance is captured by the \emph{total variational distance} $\Delta(D_0,D_1)=\frac{1}{2}\sum_x |\Pr[x\gets D_0]-\Pr[x\gets D_1]|$. 

\begin{itemize}
    \item {\bf Part (a), 20 points.} Prove the following: suppose we choose a random bit $b$, and then sample $x\gets D_b$ and apply some procedure $P$ to make a guess $b'$. Define $\epsilon(P)$ such that $\Pr[b'=b]=\frac{1+\epsilon}{2}$. Prove that $\Delta(D_0,D_1)$ is the maximum over all possible (potentially inefficient) procedures $P$ of $|\epsilon(P)|$. This contains two parts: (1) show that any procedure has $|\epsilon(P)|\leq\Delta(D_0,D_1)$, and show that (2) there exists some potentially inefficient procedure such that $\epsilon(P)=\Delta(D_0,D_1)$. For simplicity, you may assume the procedures are deterministic.
\end{itemize}
Thus, $\Delta(D_0,D_1)=0$ means that no algorithm can do better than random guessing, while $\Delta(D_0,D_1)=1$ means it is possible to perfectly distinguish the two distributions.

The way to quantify the distance between two mixed states represented by density matrices $\rho_0,\rho_1$ is through the trace distance. The trace distance has different notations throughout the literature, but is often denoted $\|\rho_0-\rho_1\|_1$. It is defined as follows: Let $\lambda_1,\cdots,\lambda_n$ be the eigenvalues of $\rho_0-\rho_1$. Then $\|\rho_0-\rho_1\|_1=\sum_i |\lambda_i|$.

\begin{itemize}
    \item {\bf Part (b), 10 points.} Consider some process for distinguishing $\rho_0$ from $\rho_1$. For simplicity, assume the process simply applies a unitary $U$, and then measures to get a string $x$. Call the resulting distributions over $x$ $D_0$ and $D_1$, respectively. Show that there exists a unitary $U$ such that $\Delta(D_0,D_1)=\|\rho_0-\rho_1\|_1/2$, where $D_0,D_1$ are the probabilities obtained from applying $U$ and then measuring. \emph{[Hint: Think about diagonalization.]}
\end{itemize}
It turns out that for \emph{any} unitary $U$, we have $\Delta(D_0,D_1)\leq\|\rho_0-\rho_1\|_1/2$ (though you do not need to show this). Thus, trace distance is the direct quantum analog (up to a factor of two) of total variational distance, in that it exactly captures the ability to distinguish two quantum states.



\section{Problem 3 (40 points)}

A pseudorandom state (PRS) is a collection of $2^\lambda$ states $\{|\psi_k\rangle\}_{k\in\{0,1\}^\lambda}$. Let $q$ be the number of qubits of the $|\psi_k\rangle$. The goal of a PRS is for $q>\lambda$, but for $|\psi_k\rangle$ for a random choice of $k$ to look like a truly random state. Note that the density matrix for a truly random state on $q$ qubits is $\frac{1}{2^q}\Id$, where $\Id$ is the identity matrix of dimension $2^q$.

\begin{itemize}
    \item {\bf Part (a). 20 points.} Show that for $q>\lambda$, there is an inefficient quantum attack which distinguishes $|\psi_k\rangle$ for a random $k$ from truly random. To do so, consider the density matrix $\rho$ for $|\psi_k\rangle$, and consider the possible eigenvalues of $\rho$. How many are non-zero? What does this tell you about the trace distance from $\frac{1}{2^q}\Id$? Thus, PRS's require computational assumptions
    \item {\bf Part (b). 10 points} Consider the following commitment scheme built from a PRS. To commit to 0, construct the superposition $\frac{1}{\sqrt{2^q}}\sum_{x\in\{0,1\}^q}|x\rangle|x\rangle$, and give the second register to Bob, keeping the first register for ourselves. To commit to 1, construct the superposition $\frac{1}{\sqrt{2^\lambda}}\sum_{k\in\{0,1\}^\lambda}|k\rangle|\psi_k\rangle$, and give the second register to Bob, keeping the first register for ourselves.

    Show that the scheme is computationally hiding, assuming the PRS is secure.

    \item {\bf Part (c). 10 points.} Suppose Alice has committed to 0. Explain why there is no unitary she can apply to her state that allows her to transform the joint state into a commitment to 1. This is not a full proof of statistical binding, but gives the idea.
\end{itemize}


\end{document}
